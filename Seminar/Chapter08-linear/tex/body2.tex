\section{サポートベクトル分類器の解}
この節では第2節にて述べたソフトマージン・サポートベクターマシンの解について扱う。ソフトマージン・サポートベクターマシンではマージンについての制限を緩和し、以下のようとなる。
\begin{equation}
  (\bm{w}^{\mathrm{T}}\bm{x_i} + \gamma) \ge 1 - \xi_i
\end{equation}
ただし、$\xi_i \ge 0\text{\ \ \ },\forall i = 0, 1, ..., n$

つまり、サポートベクターが定めた超平面を破るようなサンプルがあってもそれを認めることで制限緩和を実現している。
制限緩和することにより、マージンを最大化する問題は以下のように書き換えられる。
\begin{equation}
  \min\frac{1}{2}\|\bm{w}\| + C\sum_{i=1}^{n}\xi_i\text{\ \ \ }s.t.\text{\ \ \ }(\bm{w}^{\mathrm{T}}\bm{x_i} + \gamma)y_i \ge 1 - \xi_i, \xi_i \ge 0
\end{equation}
この問題の解を求めるためにラグランジュ未定乗数法を用いる。その際のラグランジュ関数を以下のように置く。
\begin{equation}
  \begin{split}
    {\it{L}}(\bm{w}, \gamma, \bm{\xi}, \bm{\alpha}, \bm{\beta}) = \
    \frac{1}{2}\|\bm{w}\| + C\sum_{i=1}^{n}\xi_i - \\
    $\sum_{i=1}^{n}\alpha_i\{(\bm{w}^{\mathrm{T}}\bm{x_i} + \gamma)y_i -1 + \xi_i\} - \sum_{i=1}^{n}\beta_i\xi_i
  \end{split}
\end{equation}
故に求めたい問題は、
\begin{equation}
  \min_{\bm{w}, \gamma, \xi}\max_{\alpha, \beta}{\it{L}}(\bm{w}, \gamma, \bm{\xi}, \bm{\alpha}, \bm{\beta})\text{\ \ \ }s.t.\text{\ \ \ }\bm{\alpha} \ge \bm{0}, \bm{\beta} \ge \bm{0}
\end{equation}
に書き換えられる。
$\min_{\bm{w}, \gamma, \xi}{\it{L}}(\bm{w}, \gamma, \bm{\xi}, \bm{\alpha}, \bm{\beta})$の最適性条件は、
\begin{eqnarray}
  \frac{\partial \it{L}}{\partial \bm{w}} = \bm{0} \\ 
  \frac{\partial \it{L}}{\partial \bm{\gamma}} = 0 \\
  \frac{\partial \ib{L}}{\partial \bm{\xi}} = \bm{0} 
\end{eqnarray}
すなわち、
\begin{eqnarray}
  \bm{w} = \sum_{i=1}^{n}\alpha_iy_i\bm{x_i} \\
  \sum_{i=1}^{n}\alpha_iy_i = 0 \\
  \bm{\alpha} + \bm{\beta} = \bm{C} 
\end{eqnarray}
この結果により、$\bm{w}$は$\bm{\alpha}$によって書き換えられ、$\bm{\alpha} + \bm{\beta} = \bm{C}$により$\bm{\xi}$と$\bm{\gamma}$により消去できることが分かる。
したがって元の問題は以上の結果を用いると以下のように書き換えられる。
\begin{equation}
  \max_{\bm{\alpha}}{\it{L}}(\bm{\alpha})\text{\ \ \ }s.t.\text{\ \ \ }\bm{0}\le\bm{\alpha}\le\bm{C}, \sum_{i=1}^{n}\alpha_iy_i = 0
\end{equation}
ただし、${\it{L}}(\bm{\alpha}) = \sum_{i=1}^{n}\alpha_i - \frac{1}{2}\sum_{i,j=1}^{n}\alpha_i\alpha_jy_iy_j\bm{x_i}^{\mathrm{T}}\bm{x_j}^{\mathrm{T}}$ \\
故に、
\begin{equation}
  \bm{\hat{\alpha}} = \argmax_{\bm{\alpha}}{\it{L}}(\bm{\alpha})\text{\ \ \ }s.t.\text{\ \ \ }\bm{0}\le\bm{\alpha}\le\bm{C}, \sum_{i=1}^{n}\alpha_iy_i = 0
\end{equation}
以上のような$\bm{\hat{\alpha}}$を求めることができれば、そこから$\bm{\hat{w}}$と$\hat{\gamma}$を算出できる。$\bm{\hat{\alpha}}$についての最適化問題は二次計画問題に分類される。二次計画問題については\url{http://www.support-vector-machines.org/}にて公開されているソフトウェアを用いて解くことができる。なお、$\hat{\gamma}$の求め方については次節にて紹介する。
    
    


  
 

  
  

